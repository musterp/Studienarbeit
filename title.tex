% pdf-Infos setzen
\hypersetup{%
    colorlinks=false, 
    pdfborder={0 0 0},
	pdftitle={Ferroelektrische Schichten als sensitive Materialien für feldeffektbasierte pH-Sensoren},
    pdfsubject={Untersuchung ferro- und paraelektrischer BaTiO3- und SrTiO3-Schichten},
    pdfauthor={Pascal Muster},
    pdfkeywords={pH-Sensoren, EIS-Strukturen, Feldeffekt, ISFET, Sensorik, Elektronik}
}

% Title Options
\TUBAFFakultaet{Fakultät für Werkstoffwissenschaft und Werkstofftechnologie}
\TUBAFInstitut{Institut für Elektronik- und Sensormaterialien (IESM)}
\TUBAFLehrstuhl{Arbeitsgruppe Nanofluidik}
\TUBAFZweitlogo{\includegraphics{Bilder/ESM_Logo.png}}
\TUBAFTitel%
[Ferroelektrische Schichten als sensitive Materialien für feldeffektbasierte pH-Sensoren]{Ferroelektrische Schichten als sensitive Materialien für feldeffektbasierte pH-Sensoren}
\TUBAFUntertitel{Untersuchung von \ce{SrTiO3}- und \ce{BaTiO3}-Schichten}
\TUBAFBetreuer{%
    M.\,Sc. Thomas Ihling und Dr.\,rer.\,nat. Pal Arki\newline
    Gustav-Zeuner-Straße 3, 09599 Freiberg
}
\TUBAFKorrektor{%
    Prof.\,Dr.\,rer.\,nat. Yvonne Joseph\newline
    Gustav-Zeuner-Straße 3, 09599 Freiberg
}
\TUBAFAutor[P. Muster]{Pascal Muster}
\TUBAFStudiengang[NT]{Nanotechnologie (Diplom)}
\TUBAFVertiefung{Nanoanalytik}
\TUBAFMatrikel{59\,824}
\TUBAFAnmeldedatum{2. April 2019}
\TUBAFDatum[2020-04-02]{2. April 2020}

\addto\captionsngerman{%
    \renewcommand{\TUBAFBetreuername}{Betreuer:\ }
    \renewcommand{\TUBAFKorrektorname}{Prüfer:\ }
    \renewcommand{\TUBAFAnmeldedatumname}{Angemeldet am:\ }
}

\hbadness=10000
\vbadness=10000